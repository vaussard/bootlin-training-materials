\subsection{Useful general-purpose kernel APIs}

\begin{frame}
  \frametitle{Memory/string utilities}
  \begin{itemize}
  \item In \kfile{include/linux/string.h}
    \begin{itemize}
    \item Memory-related: \kfunc{memset}, \kfunc{memcpy},
      \kfunc{memmove}, \kfunc{memscan}, \kfunc{memcmp}, \kfunc{memchr}
    \item String-related: \kfunc{strcpy}, \kfunc{strcat}, \kfunc{strcmp},
      \kfunc{strchr}, \kfunc{strrchr}, \kfunc{strlen} and variants
    \item Allocate and copy a string: \kfunc{kstrdup}, \kfunc{kstrndup}
    \item Allocate and copy a memory area: \kfunc{kmemdup}
    \end{itemize}
  \item In \kfile{include/linux/kernel.h}
    \begin{itemize}
    \item String to int conversion: \kfunc{simple_strtoul},
      \kfunc{simple_strtol}, \kfunc{simple_strtoull},
      \kfunc{simple_strtoll}
    \item Other string functions: \kfunc{sprintf}, \kfunc{sscanf}
    \end{itemize}
  \end{itemize}
\end{frame}

\begin{frame}
  \frametitle{Useful macros}
  \begin{itemize}
   \item Most useful: \kfunc{container_of}, \kfunc{ARRAY_SIZE}
   \item Bit manipulation: \kfunc{BIT}, \kfunc{BIT_MASK}, \kfunc{GENMASK}
   \item Math: \kfunc{DIV_ROUND_UP}, \kfunc{abs},...
   \item Endianness: \kfunc{cpu_to_le32}, \kfunc{cpu_to_be32},...
   \item Build-time assert: \kfunc{BUILD_BUG_ON}, \kfunc{BUILD_BUG_ON_MSG},...
   \item Run-time assert: \kfunc{WARN}, \kfunc{WARN_ON},...
   \item Kconfig: \kfunc{IS_ENABLED}, \kfunc{IS_BUILTIN}, \kfunc{IS_MODULE},...
   \item And many more, be curious !
  \end{itemize}

\end{frame}

\begin{frame}
  \frametitle{Linked lists}
  \begin{itemize}
  \item Convenient linked-list facility in \kfile{include/linux/list.h}
    \begin{itemize}
    \item Used in thousands of places in the kernel
    \end{itemize}
  \item Add a \kstruct{list_head} member to the structure whose
    instances will be part of the linked list. It is usually named
    \code{node} when each instance needs to only be part of a single
    list.
  \item Define the list with the \kfunc{LIST_HEAD} macro for a global
    list, or define a \kstruct{list_head} element and initialize
    it with \kfunc{INIT_LIST_HEAD} for lists embedded in a structure.
  \item Then use the \code{list_*()} API to manipulate the list
    \begin{itemize}
    \item Add elements: \kfunc{list_add}, \kfunc{list_add_tail}
    \item Remove, move or replace elements: \kfunc{list_del},
      \kfunc{list_move}, \kfunc{list_move_tail},
      \kfunc{list_replace}
    \item Test the list: \kfunc{list_empty}
    \item Iterate over the list: \code{list_for_each_*()} family of macros
    \end{itemize}
  \end{itemize}
\end{frame}

\begin{frame}[fragile]
  \frametitle{Linked lists examples 1/2}
  From \kfile{include/soc/at91/atmel_tcb.h}
\begin{minted}{c}
/*
 * Definition of a list element, with a
 * struct list_head member
 */
struct atmel_tc
{
    /* some members */
    struct list_head node;
};
\end{minted}
\end{frame}

\begin{frame}[fragile]
  \frametitle{Linked lists examples 2/2}
  From \kfile{drivers/misc/atmel_tclib.c}
\begin{minted}[fontsize=\scriptsize]{c}
/* Define the global list */
static LIST_HEAD(tc_list);

static int __init tc_probe(struct platform_device *pdev) {
    struct atmel_tc *tc;
    tc = kzalloc(sizeof(struct atmel_tc), GFP_KERNEL);
    /* Add an element to the list */
    list_add_tail(&tc->node, &tc_list);
}

struct atmel_tc *atmel_tc_alloc(unsigned block, const char *name)
{
    struct atmel_tc *tc;
    /* Iterate over the list elements */
    list_for_each_entry(tc, &tc_list, node) {
        /* Do something with tc */
    }
    [...]
}
\end{minted}
\end{frame}
